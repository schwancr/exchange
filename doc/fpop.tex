\documentclass[twocolumn,floatfix,nofootinbib,aps]{revtex4-1}
\usepackage[utf8]{inputenc}

\usepackage{amsmath}    % need for subequations
\usepackage{amssymb}    % for symbols
\usepackage{graphicx}   % need for figures
\usepackage{verbatim}   % useful for program listings
\usepackage{color}      % use if color is used in text
\usepackage{subfigure}  % use for side-by-side figures
%\usepackage{hyperref}   % use for hypertext links, including those to external documents and URLs
\usepackage[capitalise]{cleveref}   % use for referencing figures/equations
\begin{document}

\title{Predicting FPOP Experimental Observables within the MSM Framework}
\author{Christian R. Schwantes}
\author{Diwakar Shukla}
\author{Vijay S. Pande}

\begin{abstract}
Framework for predicting FPOP experiments from folding simulations and Markov State Models.\end{abstract}

\maketitle

To model the FPOP experiment within the MSM framework, we have to model the exchange reaction probabilistically. First, we need to define some things from the MSM framework.

\begin{enumerate}
\item $T$ : transition probability matrix, $T_{ij}$ is the probability of transferring to state $j$ in time $\tau$ (the lag time) given that the current state is state $i$
\item $P_{ij}$ : (relative) population of protein that is in state $i$ in the MSM and is currently unlabeled at residue $j$
\item $M_{ij}$ : total population of protein that is in state $i$ and is labeled at state $j$
\item $k_{ij}$ : rate constant of reaction with radical hydroxyl. It is both a function of the current state (i.e. its SASA) as well as the residue's identity.


The reactions that govern the exchange reaction can be modeled with the following ODE's:
\begin{equation}
\frac{d(M_ij)}{dt} = k_{ij} P_{ij} * R_0 OH
\end{equation}

\begin{equation}
\frac{d(OH)}{dt} = - \sum_{i, j} k_{ij} P_{ij} * R_0 OH 


\bibliography{bibliography}
\end{document}
